% Options for packages loaded elsewhere
\PassOptionsToPackage{unicode}{hyperref}
\PassOptionsToPackage{hyphens}{url}
%
\documentclass[
]{article}
\usepackage{lmodern}
\usepackage{amsmath}
\usepackage{ifxetex,ifluatex}
\ifnum 0\ifxetex 1\fi\ifluatex 1\fi=0 % if pdftex
  \usepackage[T1]{fontenc}
  \usepackage[utf8]{inputenc}
  \usepackage{textcomp} % provide euro and other symbols
  \usepackage{amssymb}
\else % if luatex or xetex
  \usepackage{unicode-math}
  \defaultfontfeatures{Scale=MatchLowercase}
  \defaultfontfeatures[\rmfamily]{Ligatures=TeX,Scale=1}
\fi
% Use upquote if available, for straight quotes in verbatim environments
\IfFileExists{upquote.sty}{\usepackage{upquote}}{}
\IfFileExists{microtype.sty}{% use microtype if available
  \usepackage[]{microtype}
  \UseMicrotypeSet[protrusion]{basicmath} % disable protrusion for tt fonts
}{}
\makeatletter
\@ifundefined{KOMAClassName}{% if non-KOMA class
  \IfFileExists{parskip.sty}{%
    \usepackage{parskip}
  }{% else
    \setlength{\parindent}{0pt}
    \setlength{\parskip}{6pt plus 2pt minus 1pt}}
}{% if KOMA class
  \KOMAoptions{parskip=half}}
\makeatother
\usepackage{xcolor}
\IfFileExists{xurl.sty}{\usepackage{xurl}}{} % add URL line breaks if available
\IfFileExists{bookmark.sty}{\usepackage{bookmark}}{\usepackage{hyperref}}
\hypersetup{
  pdftitle={Básicos de R},
  pdfauthor={Armando Valdés},
  hidelinks,
  pdfcreator={LaTeX via pandoc}}
\urlstyle{same} % disable monospaced font for URLs
\usepackage[margin=1in]{geometry}
\usepackage{color}
\usepackage{fancyvrb}
\newcommand{\VerbBar}{|}
\newcommand{\VERB}{\Verb[commandchars=\\\{\}]}
\DefineVerbatimEnvironment{Highlighting}{Verbatim}{commandchars=\\\{\}}
% Add ',fontsize=\small' for more characters per line
\usepackage{framed}
\definecolor{shadecolor}{RGB}{248,248,248}
\newenvironment{Shaded}{\begin{snugshade}}{\end{snugshade}}
\newcommand{\AlertTok}[1]{\textcolor[rgb]{0.94,0.16,0.16}{#1}}
\newcommand{\AnnotationTok}[1]{\textcolor[rgb]{0.56,0.35,0.01}{\textbf{\textit{#1}}}}
\newcommand{\AttributeTok}[1]{\textcolor[rgb]{0.77,0.63,0.00}{#1}}
\newcommand{\BaseNTok}[1]{\textcolor[rgb]{0.00,0.00,0.81}{#1}}
\newcommand{\BuiltInTok}[1]{#1}
\newcommand{\CharTok}[1]{\textcolor[rgb]{0.31,0.60,0.02}{#1}}
\newcommand{\CommentTok}[1]{\textcolor[rgb]{0.56,0.35,0.01}{\textit{#1}}}
\newcommand{\CommentVarTok}[1]{\textcolor[rgb]{0.56,0.35,0.01}{\textbf{\textit{#1}}}}
\newcommand{\ConstantTok}[1]{\textcolor[rgb]{0.00,0.00,0.00}{#1}}
\newcommand{\ControlFlowTok}[1]{\textcolor[rgb]{0.13,0.29,0.53}{\textbf{#1}}}
\newcommand{\DataTypeTok}[1]{\textcolor[rgb]{0.13,0.29,0.53}{#1}}
\newcommand{\DecValTok}[1]{\textcolor[rgb]{0.00,0.00,0.81}{#1}}
\newcommand{\DocumentationTok}[1]{\textcolor[rgb]{0.56,0.35,0.01}{\textbf{\textit{#1}}}}
\newcommand{\ErrorTok}[1]{\textcolor[rgb]{0.64,0.00,0.00}{\textbf{#1}}}
\newcommand{\ExtensionTok}[1]{#1}
\newcommand{\FloatTok}[1]{\textcolor[rgb]{0.00,0.00,0.81}{#1}}
\newcommand{\FunctionTok}[1]{\textcolor[rgb]{0.00,0.00,0.00}{#1}}
\newcommand{\ImportTok}[1]{#1}
\newcommand{\InformationTok}[1]{\textcolor[rgb]{0.56,0.35,0.01}{\textbf{\textit{#1}}}}
\newcommand{\KeywordTok}[1]{\textcolor[rgb]{0.13,0.29,0.53}{\textbf{#1}}}
\newcommand{\NormalTok}[1]{#1}
\newcommand{\OperatorTok}[1]{\textcolor[rgb]{0.81,0.36,0.00}{\textbf{#1}}}
\newcommand{\OtherTok}[1]{\textcolor[rgb]{0.56,0.35,0.01}{#1}}
\newcommand{\PreprocessorTok}[1]{\textcolor[rgb]{0.56,0.35,0.01}{\textit{#1}}}
\newcommand{\RegionMarkerTok}[1]{#1}
\newcommand{\SpecialCharTok}[1]{\textcolor[rgb]{0.00,0.00,0.00}{#1}}
\newcommand{\SpecialStringTok}[1]{\textcolor[rgb]{0.31,0.60,0.02}{#1}}
\newcommand{\StringTok}[1]{\textcolor[rgb]{0.31,0.60,0.02}{#1}}
\newcommand{\VariableTok}[1]{\textcolor[rgb]{0.00,0.00,0.00}{#1}}
\newcommand{\VerbatimStringTok}[1]{\textcolor[rgb]{0.31,0.60,0.02}{#1}}
\newcommand{\WarningTok}[1]{\textcolor[rgb]{0.56,0.35,0.01}{\textbf{\textit{#1}}}}
\usepackage{graphicx}
\makeatletter
\def\maxwidth{\ifdim\Gin@nat@width>\linewidth\linewidth\else\Gin@nat@width\fi}
\def\maxheight{\ifdim\Gin@nat@height>\textheight\textheight\else\Gin@nat@height\fi}
\makeatother
% Scale images if necessary, so that they will not overflow the page
% margins by default, and it is still possible to overwrite the defaults
% using explicit options in \includegraphics[width, height, ...]{}
\setkeys{Gin}{width=\maxwidth,height=\maxheight,keepaspectratio}
% Set default figure placement to htbp
\makeatletter
\def\fps@figure{htbp}
\makeatother
\setlength{\emergencystretch}{3em} % prevent overfull lines
\providecommand{\tightlist}{%
  \setlength{\itemsep}{0pt}\setlength{\parskip}{0pt}}
\setcounter{secnumdepth}{-\maxdimen} % remove section numbering
\ifluatex
  \usepackage{selnolig}  % disable illegal ligatures
\fi

\title{Básicos de R}
\author{Armando Valdés}
\date{22/1/2021}

\begin{document}
\maketitle

\hypertarget{temario-de-la-sesiuxf3n-1}{%
\subsection{Temario de la sesión 1}\label{temario-de-la-sesiuxf3n-1}}

1.-Conociendo el entorno de R Studio 2.-Algunos atajos del teclado
3.-Importación de archivos CSV

-Desde una ruta

-Desde un directorio

-Desde navegador

-Desde la nube

3.1.- Importar archivos xlsx y librerias 4.-Inspeccionar datos
5.-Manipular datos 6.-Regresión Lineal Simple

\hypertarget{conociendo-el-entorno-de-r-studio}{%
\subsubsection{\texorpdfstring{1.-Conociendo el entorno de R Studio
}{1.-Conociendo el entorno de R Studio  }}\label{conociendo-el-entorno-de-r-studio}}

\begin{Shaded}
\begin{Highlighting}[]
\CommentTok{\# View{-}{-} panes {-}{-}pane layout}
\CommentTok{\# Diferencias entre consola y fuente}
\NormalTok{a}\OtherTok{=}\DecValTok{10}
\NormalTok{b}\OtherTok{=}\DecValTok{15}
\NormalTok{a}\SpecialCharTok{+}\NormalTok{b}
\end{Highlighting}
\end{Shaded}

\begin{verbatim}
## [1] 25
\end{verbatim}

\begin{Shaded}
\begin{Highlighting}[]
\NormalTok{a}\OtherTok{\textless{}{-}}\DecValTok{10}
\NormalTok{b}\OtherTok{\textless{}{-}}\DecValTok{15}
\NormalTok{a}\SpecialCharTok{+}\NormalTok{b}
\end{Highlighting}
\end{Shaded}

\begin{verbatim}
## [1] 25
\end{verbatim}

\begin{Shaded}
\begin{Highlighting}[]
\NormalTok{c}\OtherTok{\textless{}{-}}\NormalTok{a}\SpecialCharTok{+}\NormalTok{b}
\FunctionTok{print}\NormalTok{(c)}
\end{Highlighting}
\end{Shaded}

\begin{verbatim}
## [1] 25
\end{verbatim}

\begin{Shaded}
\begin{Highlighting}[]
\NormalTok{a}\OtherTok{\textless{}{-}}\DecValTok{10}
\NormalTok{b}\OtherTok{\textless{}{-}}\DecValTok{15}
\NormalTok{c}\OtherTok{\textless{}{-}}\NormalTok{a}\SpecialCharTok{+}\NormalTok{b}
\FunctionTok{print}\NormalTok{(c)}
\end{Highlighting}
\end{Shaded}

\begin{verbatim}
## [1] 25
\end{verbatim}

\hypertarget{algunos-atajos-del-teclado}{%
\subsubsection{\texorpdfstring{2.-Algunos atajos del teclado
}{2.-Algunos atajos del teclado  }}\label{algunos-atajos-del-teclado}}

\begin{Shaded}
\begin{Highlighting}[]
\CommentTok{\# Comentarios}
\CommentTok{\# ctrl + shif + c   Comentario multilinea}
\CommentTok{\# ctrl + enter Ejecutar código seleccionado}
\CommentTok{\# ctrl + l Limpiar la consola}
\CommentTok{\# ctrl + s Guardar}
\CommentTok{\# alt 126 \textasciitilde{}}
\CommentTok{\# alt +91 +93 []}
\CommentTok{\# alt +123  125 \{\}}
\CommentTok{\# alt +38 \&}
\CommentTok{\# alt +124 |}
\end{Highlighting}
\end{Shaded}

\hypertarget{importacion-de-archivos-csv}{%
\subsubsection{\texorpdfstring{3.-Importacion de archivos
CSV}{3.-Importacion de archivos CSV }}\label{importacion-de-archivos-csv}}

\begin{Shaded}
\begin{Highlighting}[]
\CommentTok{\# Importar Archivos desde ruta}
\NormalTok{datos1}\OtherTok{=} \FunctionTok{read.csv}\NormalTok{(}\StringTok{"C:/Users/Armando/Documents/borrar/iris.csv"}\NormalTok{)}
\CommentTok{\#print(datos1)}
\CommentTok{\#View(datos1)}
\end{Highlighting}
\end{Shaded}

\begin{Shaded}
\begin{Highlighting}[]
\CommentTok{\# Importar archivo desde directorio}

\CommentTok{\# ver directorio}
\CommentTok{\#getwd()}

\CommentTok{\# Cambiar directorio}
\CommentTok{\#setwd("C:/Users/Armando/Desktop/Proyectos Github/basicos\_r\_parte\_I")}

\CommentTok{\# Estando en el directorio solo se llama por el nombre del archivo}
\CommentTok{\#datos1=read.csv("iris.csv")}

\CommentTok{\#View(datos1)}
\end{Highlighting}
\end{Shaded}

\begin{Shaded}
\begin{Highlighting}[]
\CommentTok{\# Importar archivos desde navegador windows}

\CommentTok{\#datos1 \textless{}{-} read.csv(file.choose(), stringsAsFactors = TRUE)}
\CommentTok{\#View(datos1)}
\end{Highlighting}
\end{Shaded}

\begin{Shaded}
\begin{Highlighting}[]
\CommentTok{\# Importar desde la nube}

\NormalTok{myurl}\OtherTok{=}\StringTok{"https://docs.google.com/spreadsheets/d/1\_3FfbYFKWrUQWb8inaFe6D2uharQfAbcPXDGVmXERSY/pub?gid=0\&single=true\&output=csv"}
\NormalTok{datos1}\OtherTok{\textless{}{-}}\FunctionTok{read.csv}\NormalTok{(}\FunctionTok{url}\NormalTok{(myurl))}
\CommentTok{\#View(datos1)}
\end{Highlighting}
\end{Shaded}

\begin{Shaded}
\begin{Highlighting}[]
\CommentTok{\# Obtener links desde la nube}
\CommentTok{\# Importar una Spreadsheet de Google}
\CommentTok{\# 1.{-}Se da click en Archivo{-}\textgreater{} Publicar en la web (no se toma el link que da)}
\CommentTok{\# 2.{-}Se copia este link y se reemplaza por el key del original (puede ser csv o xlsx)}
\CommentTok{\# https://docs.google.com/spreadsheets/d/KEY/pub?gid=0\&single=true\&output=csv}
\CommentTok{\# https://docs.google.com/spreadsheets/d/KEY/pub?gid=0\&single=true\&output=xlsx}


\CommentTok{\# Traer archivos desde DropBox}
\CommentTok{\# 1.{-}Darle click en compartir}
\CommentTok{\# 2.{-}Darle click en copiar vínculo}
\CommentTok{\# 3.{-}Cambiar "dl=0" por "dl=1"}
\end{Highlighting}
\end{Shaded}

\hypertarget{importar-archivos-xlsx-y-librerias}{%
\subsubsection{\texorpdfstring{3.1.- Importar archivos xlsx y librerias
}{3.1.- Importar archivos xlsx y librerias  }}\label{importar-archivos-xlsx-y-librerias}}

\begin{Shaded}
\begin{Highlighting}[]
\CommentTok{\# Instalar la biblioteca}
\CommentTok{\#install.packages(\textquotesingle{}readxl\textquotesingle{}) \#Se Instala 1 sola vez}

\CommentTok{\# Asi se llama a la biblioteca}
\CommentTok{\#library(readxl)}

\CommentTok{\# Llamamos la base igual que las veces anteriores (read\_excel por read.csv)}
\CommentTok{\#Datos1\textless{}{-}read\_excel(file.choose())}
\end{Highlighting}
\end{Shaded}

\hypertarget{inspeccionar-datos}{%
\subsubsection{\texorpdfstring{4.-Inspeccionar datos
}{4.-Inspeccionar datos  }}\label{inspeccionar-datos}}

\begin{Shaded}
\begin{Highlighting}[]
\CommentTok{\# Inspeccionar datos}

\CommentTok{\#print(datos1)}
\CommentTok{\#View(head(datos1))}
\FunctionTok{head}\NormalTok{(datos1,}\DecValTok{7}\NormalTok{)}
\end{Highlighting}
\end{Shaded}

\begin{verbatim}
##   Tratamiento Presion
## 1           a     180
## 2           a     173
## 3           a     175
## 4           a     182
## 5           a     181
## 6           b     172
## 7           b     158
\end{verbatim}

\begin{Shaded}
\begin{Highlighting}[]
\FunctionTok{tail}\NormalTok{(datos1)}
\end{Highlighting}
\end{Shaded}

\begin{verbatim}
##    Tratamiento Presion
## 20           d     155
## 21           e     147
## 22           e     152
## 23           e     143
## 24           e     155
## 25           e     160
\end{verbatim}

\begin{Shaded}
\begin{Highlighting}[]
\FunctionTok{summary}\NormalTok{(datos1)}
\end{Highlighting}
\end{Shaded}

\begin{verbatim}
##  Tratamiento           Presion     
##  Length:25          Min.   :143.0  
##  Class :character   1st Qu.:158.0  
##  Mode  :character   Median :162.0  
##                     Mean   :163.7  
##                     3rd Qu.:172.0  
##                     Max.   :182.0
\end{verbatim}

\begin{Shaded}
\begin{Highlighting}[]
\FunctionTok{str}\NormalTok{(datos1)}
\end{Highlighting}
\end{Shaded}

\begin{verbatim}
## 'data.frame':    25 obs. of  2 variables:
##  $ Tratamiento: chr  "a" "a" "a" "a" ...
##  $ Presion    : int  180 173 175 182 181 172 158 167 160 175 ...
\end{verbatim}

\hypertarget{manipular-datos}{%
\subsubsection{\texorpdfstring{5.-Manipular datos
}{5.-Manipular datos  }}\label{manipular-datos}}

\begin{Shaded}
\begin{Highlighting}[]
\NormalTok{myurl}\OtherTok{=}\StringTok{"https://www.dropbox.com/s/mp2g6s81b29rlha/atitanic.csv?dl=1"}
\NormalTok{datos7}\OtherTok{\textless{}{-}}\FunctionTok{read.csv}\NormalTok{(}\FunctionTok{url}\NormalTok{(myurl))}
\CommentTok{\#View(datos7)}
\end{Highlighting}
\end{Shaded}

\begin{Shaded}
\begin{Highlighting}[]
\CommentTok{\# Nuevo data Frame solo edad y sexo}
\NormalTok{Datos8}\OtherTok{\textless{}{-}}\NormalTok{ datos7[,}\FunctionTok{c}\NormalTok{(}\DecValTok{3}\NormalTok{,}\DecValTok{5}\NormalTok{)]}
\CommentTok{\#View(Datos8)}
\end{Highlighting}
\end{Shaded}

\begin{Shaded}
\begin{Highlighting}[]
\CommentTok{\# También se pueden especificar las variables por nombre}
\CommentTok{\# Igual se pueden reordenar}
\NormalTok{Datos9}\OtherTok{\textless{}{-}}\NormalTok{ datos7[,}\FunctionTok{c}\NormalTok{(}\StringTok{"Age"}\NormalTok{,}\StringTok{"Sex"}\NormalTok{)]}
\CommentTok{\#View(Datos9)}
\end{Highlighting}
\end{Shaded}

\begin{Shaded}
\begin{Highlighting}[]
\CommentTok{\# Eliminar columnas}
\NormalTok{Datos10}\OtherTok{\textless{}{-}}\NormalTok{datos7[,}\FunctionTok{c}\NormalTok{(}\SpecialCharTok{{-}}\DecValTok{3}\NormalTok{,}\SpecialCharTok{{-}}\DecValTok{5}\NormalTok{)]}
\CommentTok{\#View(Datos10)}

\CommentTok{\# Eliminar columna}
\NormalTok{Datos10}\SpecialCharTok{$}\NormalTok{EPassengerId}\OtherTok{\textless{}{-}}\ConstantTok{NULL}
\NormalTok{Datos10}\SpecialCharTok{$}\NormalTok{Name}\OtherTok{\textless{}{-}}\ConstantTok{NULL}
\end{Highlighting}
\end{Shaded}

\begin{Shaded}
\begin{Highlighting}[]
\CommentTok{\# Seleccionar filas}
\NormalTok{Datos11}\OtherTok{\textless{}{-}}\NormalTok{ datos7[}\DecValTok{5}\SpecialCharTok{:}\DecValTok{9}\NormalTok{,]}
\CommentTok{\#View(Datos11)}
\end{Highlighting}
\end{Shaded}

\begin{Shaded}
\begin{Highlighting}[]
\CommentTok{\# Filtrar por condiciones}
\NormalTok{Datos12}\OtherTok{\textless{}{-}}\NormalTok{ datos7[datos7}\SpecialCharTok{$}\NormalTok{Age}\SpecialCharTok{\textgreater{}=}\DecValTok{40}\NormalTok{,]}
\NormalTok{Datos13}\OtherTok{\textless{}{-}}\NormalTok{ datos7[datos7}\SpecialCharTok{$}\NormalTok{Age}\SpecialCharTok{\textgreater{}=}\DecValTok{40}\SpecialCharTok{\&}\NormalTok{datos7}\SpecialCharTok{$}\NormalTok{Sex}\SpecialCharTok{==}\StringTok{"male"}\NormalTok{,]}
\NormalTok{Datos14}\OtherTok{\textless{}{-}}\NormalTok{ datos7[datos7}\SpecialCharTok{$}\NormalTok{Age}\SpecialCharTok{\textgreater{}=}\DecValTok{50}\SpecialCharTok{|}\NormalTok{datos7}\SpecialCharTok{$}\NormalTok{Sex}\SpecialCharTok{==}\StringTok{"male"}\NormalTok{,]}
\CommentTok{\#View(Datos14)}
\end{Highlighting}
\end{Shaded}

\hypertarget{regresiuxf3n-lineal-simple}{%
\subsubsection{\texorpdfstring{6.-Regresión Lineal Simple
}{6.-Regresión Lineal Simple  }}\label{regresiuxf3n-lineal-simple}}

\begin{Shaded}
\begin{Highlighting}[]
\CommentTok{\# Instalando las bibliotecas}
\CommentTok{\#install.packages("lmtest")}
\CommentTok{\#install.packages("readxl")}
\CommentTok{\#install.packages("ggplot2")}

\CommentTok{\# Lammamos las bibliotecas}
\CommentTok{\# library(readxl)}
\CommentTok{\# library(lmtest)}
\CommentTok{\# library(ggplot2)}
\end{Highlighting}
\end{Shaded}

\begin{Shaded}
\begin{Highlighting}[]
\CommentTok{\# Buscamos el archivo}
\CommentTok{\#Datos1\textless{}{-}read\_excel(file.choose())}
\CommentTok{\# Datos1\textless{}{-}read\_excel("C:/Users/Armando/Documents/borrar/gasto\_crecimiento.xlsx")}

\CommentTok{\# Ejecutamos la regresión simple}
\CommentTok{\# regresion1\textless{}{-}lm(Tasa\_Crecimiento\textasciitilde{}Gasto\_Publico, Datos1)}
\CommentTok{\# summary(regresion1)}
\CommentTok{\# plot(Tasa\_Crecimiento \textasciitilde{} Gasto\_Publico, data=Datos1)}
\CommentTok{\# plot(regresion1)}
\end{Highlighting}
\end{Shaded}

\begin{Shaded}
\begin{Highlighting}[]
\CommentTok{\#Gráfico con ggplot}


\CommentTok{\# ggplot(Datos1, aes(x=Gasto\_Publico, y=Tasa\_Crecimiento)) +}
\CommentTok{\#   geom\_point(size=2) + }
\CommentTok{\#   geom\_smooth(method=lm, se=FALSE, fullrange=TRUE)}
\end{Highlighting}
\end{Shaded}


\end{document}
